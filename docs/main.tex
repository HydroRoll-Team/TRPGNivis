\documentclass[conference]{IEEEtran}
\usepackage{booktabs}

\begin{document}
\title{Psi CLI Language}
\author{\IEEEauthorblockN{HsiangNianian}
    \IEEEauthorblockA{\textit{Department of Computer Science} \\
        \textit{Founder of Psi}\\
        i@jyunko.cn}}
\maketitle

\begin{abstract}
    Psi is a new programming language designed with simplicity, flexibility, and performance in mind. It provides a clean and intuitive syntax that is easy to read and write, making it an excellent choice for both beginners and experienced programmers.
\end{abstract}

\section{Introduction}
This section introduces the Psi CLI language and its features.

\section{Features}
This section describes the key features of the Psi CLI language.

\subsection{Lexer Module}
The lexer module is responsible for converting source code into a sequence of tokens.
It recognizes the basic elements of the language such as identifiers, keywords, operators, and literals.

\begin{table}[htbp]
    \caption{Code Tokens and Corresponding Meanings}
    \label{table:code_tokens}
    \centering
    \begin{tabular}{@{} c|c @{}}
        \toprule
        \textbf{Token} & \textbf{Meaning} \\
        \midrule
        list           & TYPE \\
        colors         & IDENTIFIER \\
        =              & EQUALS \\
        red            & IDENTIFIER \\
        ,              & COMMA \\
        blue           & IDENTIFIER \\
        ,              & COMMA \\
        green          & IDENTIFIER \\
        \bottomrule
    \end{tabular}
\end{table}

\subsection{Parser Module}
The parser module converts the token sequence into an abstract syntax tree (AST).

\subsection{Built-in Types Module}
The built-in types module defines the built-in types of the Psi language, such as lists and dictionaries.

\subsection{Error Handling Module}
The error handling module provides mechanisms for capturing and handling errors at runtime.

\subsection{Execution Environment Module}
The execution environment module defines the execution environment of the Psi language.

\subsection{Interpreter Module}
The interpreter module executes operations based on the AST within the execution environment.

\subsection{Mathematics Foundation Module}
The mathematics foundation module provides basic mathematical functions and constants.

\subsection{Documentation Module}
The documentation module provides API interface descriptions and usage examples.

\section{Getting Started}
This section explains how to get started with the Psi CLI language.

\section{Keywords}
This section lists the keywords used in the Psi CLI language.

\section{Contribution}
This section outlines how to contribute to the Psi project.

\section{License}
This section provides information about the license of the Psi CLI language.

\section*{Acknowledgment}
The authors would like to thank...

\section*{References}
 [1] Reference 1

    [2] Reference 2

\end{document}